\chapter{Preface}

Every book has its own story, and this book is no exception. If I were to summarize the process of creating this book in one word, that word would be “improvised.” Yet the truth is that Arliz is the result of pure, persistent curiosity that has grown in my mind for years. What you are reading now could be called a technical book, a collection of personal notes, or even a journal of unanswered questions and curiosities. But I—officially—call it a \emph{book}, because it is written not only for others but for myself, as a record of my learning journey and an effort to understand more precisely the concepts that once seemed obscure and, at times, frustrating.\\
The story of Arliz began with a simple feeling: \textbf{curiosity}.  
Curiosity about what an array truly is. Perhaps for many this question seems trivial, but for me this word—encountered again and again in algorithm and data structure discussions—always raised a persistent question.\\
Every time I saw terms like \texttt{array}, \texttt{stack}, \texttt{queue}, \texttt{linked list}, \texttt{hash table}, or \texttt{heap}, I not only felt confused but sensed that something fundamental was missing. It was as if a key piece of the puzzle had been left out. The first brief, straightforward explanations I found in various sources never sufficed; they assumed you already knew exactly what an array is and why you should use it. But I was looking for the \emph{roots}. I wanted to understand from zero what an array means, how it was born, and what hidden capacities it holds.\\
That realization led me to decide:  
\emph{If I truly want to understand, I must start from zero.}\\	
There was no deeper story behind the name “Arliz” at first—just a random choice. But over time, I found a fitting expansion:
\begin{center}
	\textbf{Arliz = Arrays, Reasoning, Logic, Identity, Zero}
\end{center}
This backronym captures the essence of the book:
\begin{itemize}
	\item \textbf{Arrays:} The fundamental data structure we aim to explore from its origins.
	\item \textbf{Reasoning:}  The logical thinking behind data organization.
	\item \textbf{Logic:} The reasoning and thought processes behind how computers organize and manipulate data.
	\item \textbf{Identity:} The notion of distinguishing, indexing, and giving “identity” to elements within structures.
	\item \textbf{Zero:} The philosophical and mathematical concept of “nothing” from which all computation, counting, and indexing originate.
\end{itemize}
In other words, Arliz is not merely a random string—it signifies the core pillars that guide this journey: from the first “zero” to the very way we reason about data. You may pronounce it “Ar-liz,” “Array-Liz,” or however you like. I personally say “ar-liz.”\\
So yes, my naming process goes like this: pick a random name… and then look for a good backronym to justify it. Very scientific, I know! \\
But Arliz is not merely a technical book on data structures. In fact, \textbf{Arliz grows alongside me}. \\
Whenever I learn something I deem worth writing, I add it to this book. Whenever I feel a section could be explained better or more precisely, I revise it. Whenever a new idea strikes me—an algorithm, an exercise, or even a simple diagram to clarify a structure—I incorporate it into Arliz.\\
This means Arliz is a living project. As long as I keep learning, Arliz will remain alive.\\	
The structure of this book has evolved around a simple belief: true understanding begins with context. That’s why Arliz doesn’t start with code or syntax, but with the origins of computation itself. We begin with the earliest tools and ideas—counting stones, the abacus, mechanical gears, and early notions of logic—long before transistors or binary digits came into play. From there, we follow the evolution of computing: from ancient methods of calculation to vacuum tubes and silicon chips, from Babbage’s Analytical Engine to the modern microprocessor. Along this journey, we discover that concepts like arrays aren’t recent inventions—they are the culmination of centuries of thought about how to structure, store, and process information.\\
In writing this book, I have always tried to follow three principles:

\begin{itemize}
	\item \textbf{Simplicity of Expression:} I strive to present concepts in the simplest form possible, so they are accessible to beginners and not superficial or tedious for experienced readers.
	\item \textbf{Concept Visualization:} I use diagrams, figures, and visual examples to explain ideas that are hard to imagine, because I believe visual understanding has great staying power.
	\item \textbf{Clear Code and Pseudocode:} Nearly every topic is accompanied by code that can be easily translated into major languages like C\texttt{++}, Java, or C\#, aiming for both clarity and practicality.
\end{itemize}

An important note: many of the algorithms in Arliz are implemented by myself. I did not copy them from elsewhere, nor are they necessarily the most optimized versions. My goal has been to understand and build them from scratch rather than memorize ready-made solutions. Therefore, some may run slower than standard implementations—or sometimes even faster. For me, the process of understanding and constructing has been more important than simply reaching the fastest result.\\	
Finally, let me tell you a bit about myself:  
I am \textbf{Mahdi}. If you prefer, you can call me by my alias: \emph{Genix}. I am a student of Computer Engineering (at least at the time of writing this). I grew up with computers—from simple games to typing commands in the terminal—and I have always wondered what lies behind this screen of black and green text. There is not much you need to know about me, just that I am someone who works with computers, sometimes gives them commands, and sometimes learns from them.\\	
I hope this book will be useful for understanding concepts, beginning your learning journey, or diving deeper into data structures. \\	
Arliz is freely available. You can access the PDF, LaTeX source, and related code at:  
\begin{center}
	\url{https://github.com/m-mdy-m/Arliz}
\end{center}
In each chapter, I have included exercises and projects to aid your understanding. Please do not move on until you have completed these exercises, because true learning happens only by solving problems.\\	
I hope this book serves you well—whether for starting out, reviewing, or simply satisfying your curiosity. And if you learn something, find an error, or have a suggestion, please let me know. As I said:
\emph{This book grows with me.}