\chapter{Preface}

Every book has its origin story, and this one is no exception. If I were to capture the essence of creating this book in a single word, that word would be \textbf{curiosity}—though \emph{improvised} comes as a close second. What you hold in your hands (or view on your screen) is the result of years of persistent questioning, a journey that began with a simple yet profound realization: I didn't truly understand what an array was.\\

This might sound trivial to some. After all, arrays are fundamental to programming, covered in every computer science curriculum, explained in countless tutorials. Yet despite encountering terms like \texttt{array}, \texttt{stack}, \texttt{queue}, \texttt{linked list}, \texttt{hash table}, and \texttt{heap} repeatedly throughout my studies, I found myself increasingly frustrated by the superficial explanations typically offered. Most resources assumed you already knew what these structures fundamentally represented—their conceptual essence, their historical significance, their mathematical foundations.\\

But I wanted the \emph{roots}. I needed to understand not just how to use an array, but what it truly meant, how it came to exist, and what hidden capacities it possessed. This led me to a decisive moment:

\begin{center}
	\emph{If I truly want to understand, I must start from zero.}
\end{center}

And so began the journey that became Arliz.\\

\section*{The Name and Its Meaning}

The name "Arliz" started as a somewhat arbitrary choice—I needed a title, and it sounded right. However, as the book evolved, I discovered a fitting expansion that captures its essence:

\begin{center}
	\textbf{Arliz = Arrays, Reasoning, Logic, Identity, Zero}
\end{center}

This backronym embodies the core pillars of our exploration:
\begin{itemize}
	\item \textbf{Arrays:} The fundamental data structure we seek to understand from its origins
	\item \textbf{Reasoning:} The logical thinking behind systematic data organization
	\item \textbf{Logic:} The formal principles that govern how computers manipulate information
	\item \textbf{Identity:} The concept of distinguishing, indexing, and assigning meaning to elements within structures
	\item \textbf{Zero:} The philosophical and mathematical foundation from which all computation, counting, and indexing originate
\end{itemize}

You may pronounce it "Ar-liz," "Array-Liz," or however feels natural to you. I personally say "ar-liz," but the pronunciation matters less than the journey it represents.\\

\section*{What This Book Represents}

Arliz is not merely a technical manual on data structures, nor is it a traditional computer science textbook. Instead, it represents something more personal and, I believe, more valuable: a comprehensive exploration of understanding itself. This book grows alongside my own learning, evolving as I discover better ways to explain concepts, uncover new connections, and develop deeper insights.\\

This living nature means that Arliz is, in many ways, a conversation—between past and present understanding, between theoretical foundations and practical applications, between the author and reader. As long as I continue learning, Arliz will continue growing.\\

The structure of this book reflects a fundamental belief: genuine understanding requires context. Rather than beginning with syntax and moving to application (the typical approach), we start with the conceptual and historical foundations that make modern data structures possible. We trace the evolution of human thought about organizing information, from ancient counting methods to contemporary computing paradigms.\\

This approach serves a specific purpose: when you understand the intellectual journey that led to arrays, you develop an intuitive grasp of their behavior, limitations, and potential that no amount of syntax memorization can provide.\\

\section*{My Approach and Principles}

Throughout the writing process, I have maintained three core principles:

\begin{enumerate}
	\item \textbf{Conceptual Clarity:} Every concept is presented in its simplest form while maintaining accuracy and depth. My goal is accessibility without superficiality.
	
	\item \textbf{Visual Understanding:} Complex ideas are accompanied by diagrams, figures, and visual examples. I believe that concepts which can be visualized are concepts that can be truly understood and retained.
	
	\item \textbf{Practical Implementation:} Nearly every topic includes working code and pseudocode that can be easily adapted to major programming languages. Theory without practice is incomplete; practice without theory is fragile.
\end{enumerate}

An important disclosure: many of the algorithms and implementations in this book are my own constructions. Rather than copying optimized solutions from established sources, I have chosen to build understanding from first principles. This means some implementations may run slower than industry standards—or occasionally faster. For me, the process of understanding and constructing has been more valuable than simply achieving optimal performance.\\

This approach reflects the book's core philosophy: genuine mastery comes from understanding principles deeply enough to reconstruct solutions, not from memorizing existing ones.\\

\section*{About the Author}

I am \textbf{Mahdi}, though you may know me by my online alias: \emph{Genix}. At the time of writing, I am a Computer Engineering student, but more fundamentally, I am someone who grew up alongside computers—from simple games to terminal commands—always wondering what lies behind the screen of black and green text.\\

My relationship with computers has been one of continuous curiosity. I am someone who gives computers commands and, more importantly, learns from their responses. There is not much more you need to know about me personally, except that this book represents my attempt to understand the digital world I inhabit as completely as possible.\\

\section*{How to Use This Book}

Arliz is freely available and open source. You can access the complete PDF, LaTeX source code, and related materials at:

\begin{center}
	\url{https://github.com/m-mdy-m/Arliz}
\end{center}

Each chapter includes carefully designed exercises and projects. Please do not skip these—they are not busy work but essential components of the learning process. True understanding comes only through active engagement with concepts, through solving problems and building solutions yourself.\\

I encourage you to approach this book as a collaborative effort. If you discover errors, have suggestions for improvement, or develop insights that could benefit other readers, please share them. This book improves through community engagement, and your contributions make it more valuable for everyone.\\

\section*{A Living Document}

Finally, I want to be transparent about what you are engaging with. This is not a finished, polished product in the traditional sense. It is an evolving exploration of fundamental concepts, growing and improving as understanding deepens. You may encounter sections that could be clearer, examples that could be more intuitive, or explanations that could be more complete.\\

This is intentional. Arliz represents learning in progress, understanding in development. It invites you to participate in this process rather than simply consume its content.\\

I hope this book serves you well—whether you are beginning your journey with data structures, seeking to deepen existing knowledge, or simply satisfying intellectual curiosity. And if you learn something valuable, discover an error, or develop an insight worth sharing, I hope you will let me know.\\

\emph{After all, this book grows with all of us.}\\