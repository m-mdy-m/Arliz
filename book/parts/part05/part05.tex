%==============================================
% PART 5: PARALLELISM & SYSTEMS
%==============================================
\part{Parallelism \& Systems}
\label{part:parallelism-systems}

\begin{partintro}
\lettrine[lines=3]{M}{odern computing} is parallel. From multi-core CPUs to massive GPU clusters, arrays must be processed concurrently. This part examines parallelism at every level—threads, SIMD, GPUs, distributed systems.

\vspace{1em}
\textbf{What Makes This Different:}
\begin{itemize}[noitemsep]
    \item \textbf{Practical Concurrency:} Real APIs and programming models
    \item \textbf{Performance Engineering:} Achieving actual speedup
    \item \textbf{Distributed Scale:} From shared memory to data centers
    \item \textbf{Modern Hardware:} Multi-core, GPU, heterogeneous systems
\end{itemize}

\begin{quote}
\textit{``The free lunch is over.''}

\hfill--- \textsc{Herb Sutter}
\end{quote}
\end{partintro}

\chapter{Concurrency vs. Parallelism}
Definitions, distinctions, concurrent execution, parallel execution, when each matters.

\chapter{Parallel Performance Metrics}
Speedup, efficiency, scalability, work, span, Amdahl's law, Gustafson's law.

\chapter{Amdahl's Law}
Serial fraction limits speedup, diminishing returns, implications for algorithm design.

\chapter{Gustafson's Law}
Scaled speedup, weak scaling vs. strong scaling, problem size growth.

\chapter{Work and Span Model}
DAG representation, work (total operations), span (critical path), parallelism = work/span.

\chapter{Processes vs. Threads}
Separate address spaces, shared memory, creation overhead, context switching.

\chapter{Thread Creation and Management}
POSIX threads (pthreads), C++11 std::thread, thread creation, joining, detaching.

\chapter{Thread Pools}
Worker threads, task queues, reducing creation overhead, load distribution.

\chapter{Work Stealing}
Dynamic load balancing, task stealing from other threads, Cilk model, deque-based implementation.

\chapter{Fork-Join Parallelism}
Divide and conquer parallelism, recursive task spawning, join synchronization.

\chapter{Parallel Recursion}
Parallelizing divide-and-conquer algorithms, base case threshold, overhead management.

\chapter{OpenMP: Parallel Loops}
Parallel for directive, work sharing, scheduling (static, dynamic, guided), loop dependencies.

\chapter{OpenMP: Data Sharing Clauses}
Shared, private, firstprivate, lastprivate, reduction, default clauses.

\chapter{OpenMP: Reductions}
Reduction operations, private copies, combining results, common patterns.

\chapter{OpenMP: Task Parallelism}
Task directive, task dependencies, taskwait, taskloop, irregular parallelism.

\chapter{OpenMP: SIMD Directives}
SIMD loops, aligned clauses, vectorization hints, safelen.

\chapter{OpenMP: Memory Model}
Flush operations, memory ordering, atomic operations, thread-safe programming.

\chapter{Data Races and Race Conditions}
Concurrent read-write, write-write conflicts, non-deterministic behavior, debugging races.

\chapter{Memory Consistency Models}
Sequential consistency, total store order (TSO), weak ordering, relaxed models.

\chapter{Happens-Before Relation}
Partial ordering, synchronizes-with, inter-thread ordering, reasoning about concurrency.

\chapter{Memory Ordering in C++11}
memory_order_relaxed, memory_order_acquire, memory_order_release, memory_order_seq_cst.

\chapter{Synchronization Primitives: Mutexes}
Mutual exclusion locks, lock/unlock, critical sections, deadlock potential.

\chapter{Spinlocks vs. Mutexes}
Busy waiting vs. blocking, context switch overhead, when to use each.

\chapter{Reader-Writer Locks}
Multiple readers, exclusive writer, read-heavy workloads, fairness considerations.

\chapter{Semaphores}
Counting semaphores, binary semaphores, signaling, producer-consumer synchronization.

\chapter{Barriers}
Thread synchronization point, waiting for all threads, reusable barriers, OpenMP barrier.

\chapter{Condition Variables}
Wait/notify pattern, avoiding busy waiting, producer-consumer queues, spurious wakeups.

\chapter{Lock-Free Programming: Fundamentals}
Atomic operations, compare-and-swap (CAS), ABA problem, wait-free vs. lock-free.

\chapter{Atomic Operations}
Atomic reads/writes, fetch-and-add, compare-and-exchange, memory ordering.

\chapter{Lock-Free Stack}
CAS-based push/pop, ABA problem, tagged pointers, hazard pointers.

\chapter{Lock-Free Queue}
Michael-Scott queue, enqueue/dequeue with CAS, memory reclamation challenges.

\chapter{Memory Reclamation in Lock-Free Structures}
Hazard pointers, epoch-based reclamation, reference counting, safe memory reclamation (SMR).

\chapter{Transactional Memory: Concepts}
Atomic blocks, optimistic concurrency, conflict detection, rollback.

\chapter{Software Transactional Memory (STM)}
STM implementations, versioning, validation, commit protocol.

\chapter{Hardware Transactional Memory (HTM)}
Intel TSX, cache-based implementation, capacity limits, fallback paths.

\chapter{False Sharing}
Cache line contention, performance degradation, detecting false sharing (perf, VTune).

\chapter{Avoiding False Sharing}
Padding to cache line boundaries, alignment, data layout optimization.

\chapter{Cache Line Bouncing}
Cache coherence traffic, ping-pong effect, minimizing shared writes.

\chapter{NUMA Architecture}
Non-uniform memory access, local vs. remote memory, latency differences.

\chapter{NUMA-Aware Programming}
First-touch policy, memory binding (numactl), thread affinity, optimizing placement.

\chapter{Thread Affinity}
Pinning threads to cores, cache locality, avoiding migration, OS interfaces.

\chapter{Data Parallelism}
Same operation on multiple data, SIMD, array operations, vectorization.

\chapter{SIMD Programming: Intrinsics}
Compiler intrinsics, vector types, operations, portability challenges.

\chapter{SIMD Programming: SSE/AVX}
x86 SIMD instruction sets, 128-bit (SSE), 256-bit (AVX), 512-bit (AVX-512).

\chapter{SIMD Programming: ARM NEON}
ARM SIMD, 128-bit vectors, intrinsics, mobile/embedded optimization.

\chapter{SIMD: Alignment Requirements}
Aligned loads/stores, performance penalties for misalignment, ensuring alignment.

\chapter{SIMD: Masked Operations}
Conditional execution, AVX-512 mask registers, handling irregular arrays.

\chapter{Autovectorization}
Compiler automatic vectorization, loop dependencies, enabling vectorization.

\chapter{Vectorization: Loop Dependencies}
Data dependencies preventing vectorization, loop-carried dependencies, array privatization.

\chapter{Vectorization: Compiler Hints}
Pragmas (ivdep, vector), assume_aligned, restrict keyword, helping the compiler.

\chapter{GPU Architecture Deep Dive}
Streaming multiprocessors (SMs), CUDA cores, warp schedulers, memory hierarchy.

\chapter{GPU Execution Model: CUDA}
Kernel launches, thread hierarchy (thread, warp, block, grid), device functions.

\chapter{GPU Thread Organization}
blockIdx, threadIdx, blockDim, gridDim, mapping threads to data.

\chapter{GPU Memory Hierarchy}
Global memory, shared memory, registers, constant memory, texture memory, L1/L2 caches.

\chapter{GPU Global Memory Access Patterns}
Coalesced access, strided access, memory transactions, alignment requirements.

\chapter{GPU Shared Memory}
On-chip memory, fast access, bank conflicts, tiling for shared memory.

\chapter{GPU Bank Conflicts}
Shared memory banks, conflict-free access patterns, broadcast, padding to avoid conflicts.

\chapter{GPU Synchronization}
__syncthreads(), block-level synchronization, warp-level primitives, atomics.

\chapter{GPU Atomic Operations}
atomicAdd, atomicCAS, global vs. shared memory atomics, performance implications.

\chapter{GPU Occupancy}
Active warps per SM, register usage, shared memory usage, occupancy calculator.

\chapter{GPU Kernel Optimization}
Maximizing occupancy, minimizing divergence, memory coalescing, arithmetic intensity.

\chapter{GPU Warp Divergence}
Branch divergence, serialized execution, masking, minimizing divergence.

\chapter{GPU Register Pressure}
Register spilling, local memory, reducing register usage, occupancy tradeoff.

\chapter{GPU Streams and Concurrency}
Asynchronous execution, overlapping compute and memory transfer, stream priorities.

\chapter{GPU Multi-Stream Processing}
Concurrent kernel execution, data dependencies, stream synchronization.

\chapter{GPU Memory Transfer Optimization}
Pinned memory, async transfers, batching, minimizing host-device communication.

\chapter{GPU Unified Memory}
Automatic data migration, page faults, prefetching, simplifying memory management.

\chapter{GPU Profiling Tools}
nvprof, Nsight Compute, Nsight Systems, identifying bottlenecks.

\chapter{GPU Sorting Algorithms}
Bitonic sort, radix sort on GPU, merge sort, library implementations (Thrust).

\chapter{GPU Reduction Algorithms}
Tree-based reduction, warp-level primitives, shuffle operations, optimized patterns.

\chapter{GPU Scan (Prefix Sum)}
Blelloch scan on GPU, work-efficient implementation, bank conflict avoidance.

\chapter{GPU Matrix Operations}
Matrix multiplication, tiling, shared memory usage, cuBLAS library.

\chapter{GPU Stencil Computations}
Halo exchange, shared memory tiling, minimizing global memory access.

\chapter{Heterogeneous Computing}
CPU-GPU collaboration, task partitioning, host-device coordination.

\chapter{OpenCL Programming Model}
Platform, device, context, command queue, kernels, portability across devices.

\chapter{OpenCL Memory Model}
Global, local, private, constant memory, explicit transfers, buffer objects.

\chapter{Distributed Computing Fundamentals}
Distributed memory, message passing, latency and bandwidth, network topology.

\chapter{Message Passing Interface (MPI): Basics}
Communicators, ranks, MPI_Init, MPI_Finalize, basic communication.

\chapter{MPI Point-to-Point Communication}
Send, receive, buffering modes, blocking vs. non-blocking, message tags.

\chapter{MPI Non-Blocking Communication}
MPI_Isend, MPI_Irecv, overlapping communication and computation, MPI_Wait.

\chapter{MPI Collective Operations}
Broadcast, scatter, gather, all-to-all, reduction, barrier, synchronization.

\chapter{MPI Reductions}
MPI_Reduce, MPI_Allreduce, custom reduction operations, combining data.

\chapter{MPI Derived Datatypes}
Contiguous, vector, indexed types, describing complex data layouts.

\chapter{Distributed Array Partitioning}
Block distribution, cyclic distribution, block-cyclic, load balancing.

\chapter{Ghost Cells and Halo Exchange}
Boundary regions, stencil computations, communication patterns, optimizing exchange.

\chapter{Domain Decomposition}
1D, 2D, 3D decomposition, minimizing surface-to-volume ratio, communication overhead.

\chapter{Load Balancing in Distributed Systems}
Static vs. dynamic, work stealing across nodes, balancing irregular workloads.

\chapter{Parallel I/O}
MPI-IO, parallel file systems (Lustre, GPFS), collective I/O, avoiding bottlenecks.

\chapter{Checkpoint/Restart}
Fault tolerance, periodic checkpointing, restart from checkpoint, overhead.

\chapter{Fault Tolerance Strategies}
Replication, algorithm-based fault tolerance, detecting failures, recovery.

\chapter{MapReduce Programming Model}
Map phase, shuffle/sort, reduce phase, data-parallel programming at scale.

\chapter{MapReduce on Arrays}
Array processing patterns, sorting, filtering, aggregation, distributed arrays.

\chapter{Hadoop and HDFS}
Distributed file system, block storage, replication, data locality.

\chapter{Spark and RDDs}
Resilient distributed datasets, in-memory processing, transformations and actions.

\chapter{Spark Array Operations}
DataFrame operations, array columns, broadcast variables, distributed array processing.

\chapter{Partitioned Global Address Space (PGAS)}
Global view with local control, UPC, Chapel, X10, one-sided communication.

\chapter{Array Notation in PGAS Languages}
Distributed array declarations, affinity, local vs. remote access.

\chapter{Distributed Consensus}
Paxos, Raft, leader election, maintaining consistency, fault tolerance.

\chapter{Eventual Consistency}
Convergent data types, CRDTs, conflict resolution, distributed arrays.

\chapter{Real-Time Systems}
Hard vs. soft deadlines, WCET (worst-case execution time), predictability.

\chapter{WCET Analysis for Array Operations}
Static analysis, timing bounds, cache analysis, array access time.

\chapter{Real-Time Scheduling}
Rate monotonic scheduling (RMS), earliest deadline first (EDF), priority assignment.

\chapter{Cache Locking for Real-Time}
Locking cache lines, predictable access times, avoiding cache misses.

\chapter{Scratchpad Memory}
Software-managed memory, predictable access, explicitly controlled, vs. caches.

\chapter{Energy-Efficient Computing}
Power models, dynamic power, static power, energy-aware algorithms.

\chapter{DVFS and Power Management}
Dynamic voltage and frequency scaling, race-to-idle vs. energy proportional.

\chapter{Energy-Efficient Array Processing}
Minimizing memory access, data reuse, approximate computing for energy savings.

\chapter{Green Computing Principles}
Energy proportionality, workload consolidation, resource efficiency.

\chapter{Parallel Debugging}
Race detection, Valgrind, ThreadSanitizer, Intel Inspector, deterministic replay.

\chapter{Performance Analysis Tools}
gprof, perf, VTune, TAU, HPCToolkit, profiling parallel programs.

\chapter{Scalability Analysis}
Strong scaling, weak scaling, scalability plots, identifying bottlenecks.

\chapter{Communication vs. Computation Ratio}
Bandwidth-limited vs. compute-limited, optimizing the ratio, hiding latency.

\chapter{Parallel Algorithm Design Patterns}
Task parallelism, data parallelism, pipeline, master-worker, SPMD.