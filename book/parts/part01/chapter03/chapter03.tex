
% ==========================================
% CHAPTER 3: EGYPTIAN SYSTEMATIC KNOWLEDGE AND GEOMETRIC ARRAYS
% ==========================================

\chapter{Egyptian Systematic Knowledge and Geometric Arrays}
% Timeline: 3,100 BCE - 300 BCE
% Focus: Systematic approaches to geometry and practical mathematics

\section{Hieroglyphic Number Systems and Decimal Thinking}
% Egyptian numerals, decimal grouping, systematic representation
% Early place-value concepts without positional notation

\section{The Rhind Papyrus: Systematic Mathematical Methods}
% Ahmes' mathematical procedures, unit fractions, systematic problem-solving
% Early algorithmic thinking in practical contexts

\section{Sacred Geometry and Architectural Arrays}
% Pyramid construction, geometric planning, systematic proportions
% Grid systems and modular construction techniques

\section{Egyptian Fractions and Systematic Decomposition}
% Unit fraction systems, systematic decomposition methods
% Early concepts of mathematical representation standards