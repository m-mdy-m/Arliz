%==============================================
% PART 1: DATA REPRESENTATION
%==============================================
\part{Data Representation}
\label{part:data-representation}

\begin{partintro}
\lettrine[lines=3]{B}{efore we} can understand how arrays store elements, we must first understand how computers represent information itself. Every piece of data—numbers, text, images, instructions—exists as patterns of bits. This part explores the foundational question: how do we encode meaning into binary?

\begin{quote}
\textit{``The choice of representation is often more important than the choice of algorithm.''}

\hfill--- \textsc{Donald E. Knuth}
\end{quote}
\end{partintro}


\chapter{The Philosophy of Representation}
\label{ch:philosophy-representation}

\begin{chapterintro}
Before we can talk about arrays, before we can understand how computers store data, we need to ask a more basic question: what does it mean to \textit{represent} something? This chapter is about that question. It might seem weird to start a book about arrays with philosophy, but trust me, we need this foundation. Without understanding representation itself, we can't really understand how arrays work or why they matter.
\end{chapterintro}

\section{Why Representation Matters}

Let me start with something simple. When you see the number "5" written on paper, what are you actually looking at? You're looking at ink on paper, right? But somehow, that ink means something. It represents the concept of five-ness. This is weird when you think about it \cite{russo2018philosophy}.\\
The thing is, computers don't work with actual numbers or letters or images. They work with electrical signals - high voltage, low voltage. That's it. But somehow we can use those electrical signals to represent anything: numbers, words, pictures, music, even this book you're reading. This is representation at work.\\
\textbf{Here's the key idea}: representation is when one thing stands for another thing. The mark "5" stands for the number five. A high voltage in computer memory stands for the number 1. A pattern of bits stands for a letter. Everything in computing is representation all the way down.\\
And arrays? Arrays are just organized representations. They're a way of representing many things in order, one after another. But before we get there, we need to understand where this whole idea of representation came from.

\section{The Beginning: Counting Before Numbers}

\subsection{Ancient Counting Practices}

Humans have been representing things for a very long time. Way before writing, way before numbers as we know them, people needed to keep track of stuff. How many sheep do I have? How many days until winter? How much grain is in storage?\\
The earliest evidence we have shows people using physical objects to represent quantities. They would use stones, or sticks, or marks on bones. Each stone represents one sheep. This is called \textit{tallying} \cite{houston2023_earlyhistory}.\\
Archaeological evidence shows tally marks from over 40,000 years ago. The Lebombo bone from South Africa has 29 notches carved into it. Each notch represents... something. We don't know what exactly, but the point is: one mark equals one thing. This is representation in its most basic form.

\subsection{From Concrete to Abstract}

Something interesting happened over thousands of years. People moved from concrete representation to abstract representation \cite{coolidge2012numerosity}.\\
What's the difference?\\
\textbf{Concrete representation}: This stone represents THIS sheep. That mark represents THAT day.\\
\textbf{Abstract representation}: The symbol "5" represents the concept of five-ness, regardless of whether we're talking about sheep, days, or anything else.\\
This shift from concrete to abstract is huge. It's one of the most important developments in human thinking. And it didn't happen overnight. It took thousands of years.\\
Peter Damerow, who studied the history of mathematical thinking, argues that this abstraction happened because of practical needs \cite{damerow1999_materialculture}. People needed to manage larger and more complex quantities. They needed to trade, to build, to organize societies. Simple tally marks weren't enough anymore.

\section{Ancient Philosophy and Representation}

\subsection{Plato's Cave}

The ancient Greek philosopher Plato told a famous story about representation. It's called the Allegory of the Cave, and it goes like this \cite{plato_republic}:\\
Imagine people chained in a cave, facing a wall. Behind them is a fire, and between them and the fire, people walk carrying objects. The chained people can only see shadows on the wall. To them, those shadows are reality. They don't know about the real objects casting the shadows, or about the fire, or about the world outside the cave.\\
Plato was using this story to talk about knowledge and reality, but it's also a story about representation. The shadows represent real objects, but they're not the objects themselves. They're just representations - and not very good ones.\\
According to the article by Russo, Plato worried a lot about representation \cite{russo2018philosophy}. He thought most of what we experience are just representations (like shadows) of perfect Forms that exist in some ideal realm. A triangle you draw on paper isn't a real triangle - it's just a representation of the perfect Form of Triangle.\\
For our purposes, what matters is this: Plato understood that representations can be misleading. They can show us something without showing us everything. This is important when we think about how computers represent information. A number in computer memory represents a value, but it's not the value itself - it's just a pattern of electrical charges.

\subsection{Aristotle and Practical Representation}

Aristotle, Plato's student, had a more practical view \cite{aristotle_metaphysics}. He wasn't so worried about perfect Forms. Instead, he studied how we actually think and represent things in our minds.\\
Aristotle noticed that when we see an object, our mind creates a sort of internal image of it. This internal image represents the object. But it's not the object - it's a mental representation. Later philosophers, especially in the Middle Ages, built on this idea.

\subsection{Medieval Thought: Aquinas and Representation}

Thomas Aquinas, a medieval philosopher, continued this discussion about representation \cite{russo2018philosophy}. He was interested in how we know things through their representations in our mind.\\
Aquinas said that when you see a tree, the tree's "form" (not quite the same as Plato's Form, but similar idea) enters your mind. Your mind doesn't contain the actual tree - that would be impossible - but it contains a representation of the tree.\\
Why am I telling you about medieval philosophy in a book about arrays? Because these thinkers were wrestling with the same basic problem we face in computing: how can one thing stand for another thing? How can patterns (whether in our minds or in computer memory) represent reality?

\section{Modern Philosophy and Representation}

\subsection{Descartes and Mental Representation}

In the 1600s, René Descartes started modern philosophy. He was obsessed with certainty - what can we know for sure? He ended up focusing on mental representations - ideas in the mind.\\
Descartes believed that when we think about something, we have a mental representation of it. These representations are distinct from the things themselves. You can think about a unicorn (you have a representation of it in your mind) even though unicorns don't exist.\\
This might seem obvious now, but it was important. It established that representations can exist independently from what they represent. This is exactly what happens in computers: we have patterns of bits that represent things, whether or not those things physically exist.

\subsection{Kant and the Structure of Representation}

Immanuel Kant, writing in the late 1700s, argued that we never experience reality directly - we only experience our representations of reality \cite{kant1964critique}. Our minds structure and organize sensory input, creating representations that we then experience as "reality."\\
For Kant, representation wasn't just passive copying. It was active construction. When you see a red apple, your mind is actively organizing color data, shape data, spatial data, etc., into a unified representation: "red apple."\\
This is relevant to computing because computers also actively construct representations. When you take a digital photo, the computer doesn't capture "the image" - it constructs a representation using millions of numbers that encode color and brightness values.

\subsection{Sartre and the Imagination}

Jean-Paul Sartre, a 20th century philosopher, wrote extensively about imagination and representation \cite{sartre1940imaginary}. He argued that when we imagine something, we're creating a special kind of representation - one that marks itself as unreal.\\
When you imagine a purple elephant, you know you're imagining. Your mental representation includes this "unreality marker." This is different from perception, where representations present themselves as real.\\
According to Russo's analysis, Sartre believed that representational consciousness involves a kind of "derealization" - the imagined object is present as absent \cite{russo2018philosophy}. This is a strange idea, but it captures something important: representations can point to things that aren't there.\\
In computing, we do this all the time. A variable in a program represents a value, but the value might not exist yet. A null pointer represents the absence of a reference. Representations can encode absence as much as presence.

\section{What Is Representation, Really?}

After all this philosophy, let's try to say clearly what representation means.\\
\textbf{Representation is when one thing (the representation) stands for another thing (the represented) according to some system or convention.}\\
A few key points:\\
\textbf{1. The representation is different from what it represents.} The word "dog" is not a dog. The number "42" in computer memory is not forty-two - it's a pattern of electrical charges that we interpret as forty-two.\\
\textbf{2. Representation requires a system or convention.} Why does "5" mean five? Because we have a convention - a shared agreement - that this symbol represents this quantity. In computing, we have conventions like "high voltage = 1, low voltage = 0."\\
\textbf{3. Representations can be ambiguous.} The same representation can mean different things in different contexts. In one context, the byte "01000001" might represent the number 65. In another context, it might represent the letter "A." The representation is the same; the interpretation differs.\\
\textbf{4. Representations can be layered.} A letter is represented by a number (ASCII code). That number is represented by a pattern of bits. Those bits are represented by electrical charges. Representation all the way down, until you hit physical reality.

\section{The Abstraction Hierarchy}

Now we're getting to something really important for understanding computing: the abstraction hierarchy.

\subsection{What Is Abstraction?}

Abstraction means hiding details. When you use abstraction, you work with simplified representations that hide underlying complexity.\\
Think about driving a car. You press the gas pedal, and the car goes faster. You don't need to think about fuel injection, combustion, crankshafts, or transmission gears. Those details are abstracted away. The gas pedal is an interface - a representation of a complex system.

\subsection{Layers of Abstraction in Computing}

Computing is built on layers of abstraction, each layer representing the layer below it.\\
\textbf{Physical Layer:} Actual electrons moving in circuits. This is the only "real" layer - everything else is representation.\\
\textbf{Electrical Layer:} We represent patterns of electrons as voltage levels. High voltage = 1, low voltage = 0.\\
\textbf{Digital Layer:} We represent voltage levels as bits. A bit is just a symbol that takes value 0 or 1, but it represents an electrical state.\\
\textbf{Number Layer:} We represent quantities using patterns of bits. The pattern 00000101 represents the number 5 (in binary).\\
\textbf{Character Layer:} We represent letters and symbols using numbers. The number 65 represents the letter A (in ASCII).\\
\textbf{Data Structure Layer:} We represent organized collections of data using structures like arrays. An array represents a sequence of values.\\
\textbf{Algorithm Layer:} We represent procedures and processes using algorithms. An algorithm represents a method for solving a problem.\\
\textbf{Application Layer:} We represent user needs and tasks using applications. A word processor represents the act of writing and editing text.\\
Each layer builds on the layer below. Each layer abstracts away details of the layer below. And crucially: each layer is a form of representation.

\subsection{Why This Matters for Arrays}

Arrays live in this hierarchy. An array is:

\begin{itemize}
\item Physically: electrical charges in memory chips
\item Electrically: voltage patterns across memory cells  
\item Digitally: sequences of bits
\item At the data structure level: an organized collection representing ordered data
\end{itemize}
When we work with arrays, we're working at the data structure level of abstraction. We don't think about voltage levels. We think about slots containing values. But understanding that arrays are representations - that they exist in this hierarchy - helps us understand their properties and limitations.

\section{Information Theory Fundamentals}

\subsection{Shannon's Insight}

In 1948, Claude Shannon published a paper that changed everything \cite{shannon1948mathematical}. It was called "A Mathematical Theory of Communication," and it established information theory - the mathematics of information and representation.\\
Shannon's key insight was this: information is about reducing uncertainty.\\
Think about it this way. Before I tell you something, you're uncertain. There are many possible things I might say. When I actually tell you something, I reduce your uncertainty. The amount of information in my message is related to how much uncertainty it removes.\\
Let's use an example. Suppose I'm going to tell you about the result of a coin flip. Before I tell you, there are two possibilities: heads or tails. You're uncertain. When I say "heads," I've removed your uncertainty. I've given you one bit of information.\\
Why one bit? Because a bit is the fundamental unit of information. It's the amount of information needed to distinguish between two equally likely possibilities.

\subsection{Information and Representation}

Shannon's theory connects directly to representation. To represent something, you need enough information to distinguish it from other possibilities.\\
If I want to represent one of two things (like heads or tails), I need 1 bit.\\
If I want to represent one of four things, I need 2 bits. (00, 01, 10, 11 - four possibilities)\\
If I want to represent one of eight things, I need 3 bits.\\
The pattern: to represent $N$ equally likely possibilities, you need $\log_2(N)$ bits.\\
This is fundamental to understanding how computers represent things. Every piece of data in a computer is encoded using some number of bits. How many bits? Enough to distinguish it from other possible values.

\subsection{Encoding and Decoding}

Information theory also clarifies the relationship between representation and interpretation.\\
\textbf{Encoding} is the process of converting something into a representation. We encode the number 5 as the bit pattern 00000101.\\
\textbf{Decoding} is the process of interpreting a representation to recover what it represents. We decode the bit pattern 00000101 as the number 5.\\
Crucially, encoding and decoding require agreement on the representation scheme. If we use different schemes, communication fails. If I encode using ASCII and you decode using EBCDIC (a different character encoding), we won't understand each other.\\
This is why standards matter so much in computing. Standards are agreed-upon representation schemes. ASCII is a standard. Unicode is a standard. IEEE 754 (for floating-point numbers) is a standard. Without standards, we couldn't share data.

\subsection{Information Content}

Not all messages contain the same amount of information. Shannon showed that information content depends on probability.\\
If I tell you something you already knew was almost certain, I've given you little information. "The sun rose this morning." - Low information content, because you were already nearly certain of this.\\
If I tell you something surprising - something you thought was very unlikely - I've given you a lot of information. "It snowed in the Sahara Desert today." - High information content, because this is unexpected.\\
Shannon formalized this with the concept of entropy (information entropy, not thermodynamic entropy, though they're related). Entropy measures the average information content of messages from a source.\\
For our purposes, what matters is this: the amount of information in a representation depends on how much uncertainty it removes. This affects how we design data structures and encoding schemes.

\section{From Philosophy to Practice}

We've covered a lot of ground in this chapter, from Plato's cave to Shannon's mathematics. Let me try to tie it together and connect it to what's coming in the rest of this book.\\
\textbf{Representation is fundamental to computing.} Everything a computer does involves representation. We represent numbers, text, images, sounds - everything - as patterns of bits.\\
\textbf{Representation involves abstraction.} We build layers of abstraction, each layer representing the layer below while hiding its complexity.\\
\textbf{Representation requires conventions.} For representations to work, we need agreed-upon schemes for encoding and decoding. These are our standards and protocols.\\
\textbf{Representation has costs and limits.} Every representation is a simplification. Some information is lost or distorted. (This is why lossy compression works - we can throw away information we decide isn't important.)\\
\textbf{Understanding representation helps us understand data structures.} An array is a representation of ordered data. Understanding what that means - really means - helps us understand how arrays work and why they're designed as they are.\\
In the next chapters, we'll move from philosophy to physics. We'll see how representation actually works in physical computers. How do you represent information using electricity? How do you build circuits that process representations? How do you organize representations in memory?\\
But I hope this chapter has given you a foundation. When we talk about bits and bytes, voltages and logic gates, remember: we're always talking about representation. One thing standing for another. Shadows on the wall of Plato's cave.\\
Only now, we understand the shadows. We can analyze them, manipulate them, build entire worlds from them. That's the power of representation, and that's what makes computing possible.

\chapter{Voltage to Bits}
How transistors encode binary states, voltage levels, noise margins, signal integrity, why binary won over ternary.

\chapter{Binary Representation and Boolean Algebra}
Two-state logic, Boolean operations, truth tables, De Morgan's laws, bit as fundamental unit.

\chapter{Number Systems and Bases}
Positional notation, decimal, binary, octal, hexadecimal. Base conversion algorithms. Historical development. Why different bases matter for different purposes.

\chapter{Integer Representation: Unsigned}
Binary positional notation, range limitations, conversion algorithms, hexadecimal convenience.

\chapter{Signed Integer Representation}
Sign-magnitude, one's complement, two's complement (why it won), arithmetic operations, overflow detection.

\chapter{Integer Overflow and Wraparound Behavior}
Undefined behavior, modular arithmetic, detecting overflow, saturating arithmetic, language-specific behaviors.

\chapter{Fixed-Point Representation}
Q-format notation, scaling factors, precision-range tradeoffs, embedded systems applications, fixed-point arithmetic.

\chapter{Floating-Point Representation: IEEE 754}
Sign, exponent, mantissa encoding, normalized and denormalized numbers, single vs. double precision.

\chapter{Special Floating-Point Values}
Infinity (positive and negative), NaN (quiet and signaling), signed zero, subnormal numbers.

\chapter{Floating-Point Arithmetic Operations}
Addition, multiplication, division, rounding modes (round to nearest, toward zero, toward ±∞), fused multiply-add.

\chapter{Floating-Point Error Analysis}
Precision loss, catastrophic cancellation, machine epsilon, relative error, Kahan summation algorithm.

\chapter{Decimal Floating-Point Representation}
IEEE 754-2008 decimal formats, financial computing requirements, densely packed decimal.

\chapter{Extended Precision and Arbitrary Precision}
80-bit extended precision, quadruple precision (128-bit), arbitrary precision libraries (GMP, MPFR).

\chapter{Character Encoding: ASCII and Extensions}
7-bit ASCII, extended ASCII variants, code pages, limitations for internationalization.

\chapter{Unicode: Universal Character Encoding}
Code points, planes, combining characters, grapheme clusters, normalization forms (NFC, NFD, NFKC, NFKD).

\chapter{Unicode Transformation Formats}
UTF-8 (variable length, backward compatible), UTF-16 (surrogate pairs), UTF-32 (fixed length), BOM issues.

\chapter{Text Processing Complexities}
Grapheme vs. code point counting, case folding, locale-dependent operations, text segmentation.

\chapter{Endianness: Byte Ordering}
Big-endian vs. little-endian, historical reasons, network byte order, bi-endian systems, byte swapping.

\chapter{Cross-Platform Data Exchange}
Serialization formats, portable binary formats, protocol design considerations.

\chapter{Bitwise Operations Fundamentals}
AND, OR, XOR, NOT operations, truth tables, bit manipulation primitives.

\chapter{Bit Shifting and Rotation}
Logical shift, arithmetic shift, rotate left/right, applications to multiplication/division by powers of 2.

\chapter{Bit Manipulation Techniques}
Setting/clearing/toggling bits, bit masks, extracting bit fields, counting set bits (population count).

\chapter{Bit Packing and Flags}
Packing multiple boolean flags, bitfields in structs, union tricks, bit arrays.

\chapter{Advanced Bit Hacking}
Finding rightmost set bit, isolating bit patterns, bit reversal, Gray code conversion.

\chapter{Data Alignment Fundamentals}
Natural alignment, alignment requirements by type, misalignment penalties.

\chapter{Structure Padding and Layout}
Compiler padding rules, struct member ordering, packing pragmas, cache line awareness.

\chapter{Alignment Optimization Techniques}
Manual padding, alignment attributes, reordering for cache efficiency.

\chapter{Color Representation Models}
RGB, RGBA, HSV, HSL, CMYK, YUV, color spaces, gamma correction.

\chapter{Pixel Formats and Bit Depth}
1-bit, 8-bit indexed, 16-bit (RGB565), 24-bit, 32-bit (with alpha), HDR formats.

\chapter{Image Compression Fundamentals}
Lossless (PNG, GIF) vs. lossy (JPEG), compression ratios, quality tradeoffs.

\chapter{Video Encoding Principles}
Frame types (I, P, B), motion compensation, codecs (H.264, H.265, VP9, AV1).

\chapter{Audio Representation: Sampling Theory}
Nyquist theorem, sample rate, bit depth, quantization noise, aliasing.

\chapter{Audio Encoding Formats}
PCM (uncompressed), FLAC (lossless), MP3, AAC, Opus (lossy), perceptual coding.

\chapter{Signal Processing Representations}
Time domain, frequency domain, spectrograms, wavelet transforms.

\chapter{Pointer Representation}
How pointers are stored in memory, pointer size (32-bit vs. 64-bit), pointer tagging.

\chapter{Pointer Arithmetic and Address Calculation}
Array element addressing, struct member offsets, pointer subtraction.

\chapter{Special Pointer Values}
Null pointers, wild pointers, dangling pointers, pointer alignment requirements.

\chapter{Error Detection Codes}
Parity bits (even/odd), checksums, CRC algorithms, hash-based integrity.

\chapter{Error Correction Codes}
Hamming codes, Reed-Solomon codes, LDPC codes, convolutional codes.

\chapter{Cyclic Redundancy Checks (CRC)}
CRC polynomials, CRC-8, CRC-16, CRC-32 standards, efficient implementation using tables, applications in data integrity.

\chapter{Data Serialization Fundamentals}
Converting in-memory structures to byte streams, portability issues.

\chapter{Binary Serialization Formats}
Protocol Buffers, FlatBuffers, Cap'n Proto, MessagePack, CBOR.

\chapter{Text Serialization Formats}
JSON, XML, YAML, TOML, human-readable vs. machine-efficient tradeoffs.

\chapter{Custom Binary Protocols}
Designing efficient binary formats, alignment considerations, versioning.

\chapter{Data Compression Theory}
Information theory basics, entropy, lossless vs. lossy bounds.

\chapter{Dictionary-Based Compression}
LZ77, LZ78, LZW, Lempel-Ziv family, dictionary construction.

\chapter{Entropy Coding}
Huffman coding, arithmetic coding, asymmetric numeral systems (ANS).

\chapter{Modern Compression Algorithms}
zlib, gzip, bzip2, LZMA, Zstandard, Brotli, compression levels.

\chapter{Specialized Compression}
Run-length encoding, delta encoding, columnar compression, domain-specific methods.