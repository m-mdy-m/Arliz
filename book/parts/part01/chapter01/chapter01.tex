% ==========================================
% CHAPTER 1: THE PRIMORDIAL URGE TO COUNT AND ORDER
% ==========================================

\chapter{The Primordial Urge to Count and Order}
% Timeline: 100,000 BCE - 8,000 BCE
% Focus: Cognitive foundations of systematic thinking
\section{The Philosophy of Measurement and Human Consciousness}
% Why humans needed to count, measure, track - cognitive foundations
% The emergence of abstract thinking about quantity and order

\section{Paleolithic Counting: Bones, Stones, and Fingers}
% Ishango bone (20,000 BCE), tally systems, body-part counting
% Archaeological evidence of early systematic representation

\section{Neolithic Revolution: Agriculture and the Need for Records}
% Agricultural surplus, seasonal tracking, early inventory systems
% The transition from nomadic to systematic settlement patterns

\section{Proto-Writing and Symbolic Representation}
% Token systems (8,000 BCE), clay envelopes, early symbolic abstraction
% The cognitive leap from concrete to abstract representation