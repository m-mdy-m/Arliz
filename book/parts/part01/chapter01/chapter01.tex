% ==========================================
% CHAPTER 1: THE PRIMORDIAL URGE TO COUNT AND ORDER
% ==========================================

\chapter{The Primordial Urge to Count and Order}
% Timeline: 100,000 BCE - 8,000 BCE
% Focus: Cognitive foundations of systematic thinking

\section{The Philosophy of Measurement and Human Consciousness}
% Why humans needed to count, measure, track - cognitive foundations
% The emergence of abstract thinking about quantity and order

The fact that the human race has been able to reach the threshold of evolution can mainly be attributed to two abilities: intelligence and creativity. While the former enables us to reason, solve problems, learn from experiences, and think abstractly, the latter enables us to change our way of thinking and create new strategies to overcome the obstacles we face. It can be concluded that every aspect of our lives in relation to innovation and scientific progress can be attributed to intelligence or creativity.\\
Based on this, it can be argued that many aspects of the formation of human computational structures—including arrays—are the result of the interaction of these two capabilities: biological intelligence that makes quantitative recognition possible and creativity that transforms these recognitions into symbols and material tools. This chapter examines how these capacities emerged and transformed into cognitive and cultural tools.

\subsection{Instinct versus Representation: Cognitive Foundations of Human Distinction}

Paleontologists and anthropologists have long been searching for explanations of the roots of modernity and modern thought. These discussions usually involve concepts such as "abstraction" and "symbolic thinking" that are often presented without precise and operational definitions.\\
To understand humanity's need for structured information systems, we must first clarify the distinction between instinctual behavior and the capacity for representation.\\
William James, the prominent 19th-century psychologist, provided a clear definition of instinct:

\begin{quote}
	Instinct is usually defined as the faculty of acting in such a way as to produce certain ends, without foresight of the ends, and without previous education in the performance. That instincts, as thus defined, exist on an enormous scale in the animal kingdom, needs no proof. They are the functional correlatives of structure. With the presence of a certain organ goes, one may say, almost always a native aptitude for its use.
\end{quote}
This definition provides deep insight into the animal kingdom: instincts are hardwired responses that automatically emerge from biological structure. A spider does not need to learn web-weaving techniques—this pattern is encoded in its neural architecture. A bird does not need to be taught to fly—motor coordination naturally emerges from its physical form.\\
But does this definition apply to humans in the same way it applies to other animals? Can we claim that early humans, from birth, had instinctual abilities for counting, organizing, and creating systematic representations of quantity? The answer is decidedly negative. Early humans, unlike animals, were not limited to instinctual reactions alone. They were able to demonstrate something beyond biological response: the capacity for representation. In other words, humans not only "saw" that one prey was larger than another, but could hold this difference in their minds as a "quantity," compare it, and even transmit it to others. This capacity for mental representation and conceptual transmission is considered a turning point in human cognitive evolution.
\subsection{Number Sense: Cognitive Foundations of Quantitative Thinking}

This capacity for representation, now called "number sense," represents humanity's tendency and ability to use numbers and quantitative methods as means for establishing communication, processing, and interpreting information. This leads to the expectation that mathematics possesses a particular order. Cognitive psychology research has shown that most humans—even without formal education—can intuitively perceive the difference between two and three. Toddlers also reveal such distinctions in their simple games. But humanity's strength was that it transformed this primitive sense into an abstract tool: from "more and less" to "exactly how many?"\\
In attempting to identify a prototypical cognitive basis for the origin of modern symbolic thinking, each candidate feature must satisfy four essential criteria:

\begin{enumerate}
	\item \textbf{Innate}: Determined by factors that exist from birth and are genetically inherited
	\item \textbf{Fundamental}: Essential for basic representational structure
	\item \textbf{Shared}: Observable with other species
	\item \textbf{Bridging}: Capable of connecting the world of sensory perception to the realm of concepts and shared expression
\end{enumerate}
Anthropological and cognitive psychology research has shown that number sense possesses these characteristics.\\
This cognitive development, observed in human infants and even some animal species (such as primates, birds, and mammals), demonstrates an innate and shared characteristic. This trait appears in early human development and is genetically inherited. Moreover, this ability forms the foundation of modern symbolic thinking.\\
Most humans, even without formal education, can intuitively perceive the difference between two and three. Children also reveal such distinctions in their simple games. But humanity's strength was that it transformed this primitive sense into an abstract tool: from "more and less" to "exactly how many?" This development represents the transition from intuitive understanding to precise measurement.

\subsection{From Intuitive Understanding to Systematic Measurement}

Imagine returning to ancient times and observing a group of early humans hunting. While hunting, without needing to know mathematics, they would go toward better prey through observation. For example, when facing two deer, one fat and one thin, the natural choice would be toward the fatter animal, because more food would be available for the group's meal. From our perspective, early humans made such decisions.\\
But when two or three fat deer were fleeing from a group of early humans, would they still choose one? The answer is probably negative. They considered these creatures as a quantitative set and would say there were "two" or "three" deer that needed to be hunted.\\
While in comparing the fat and thin deer, they would select the fat one and reject the thin one. This shows that humans from the beginning had two types of quantitative thinking: qualitative comparison (fat versus thin) and absolute counting (two versus three).\\
This is where the concept of "measurement" emerges. Measurement means dividing a continuous reality—such as time, distance, or weight—into discrete units that can be counted again. When early humans learned to count nights, or link the length of seasons with changes in sky and earth, they actually took the first step in building the "language of order." This language later appeared in the form of symbols, numbers, and ultimately tables and arrays.\\
This cognitive-cultural break has deep implications that are still observable in modern data structures:

\begin{itemize}
	\item \textbf{Discretization}: Converting continuity into discrete units
	\item \textbf{Standardization}: Creating fixed and transferable units
	\item \textbf{Reproducibility}: Enabling precise reproduction of information
\end{itemize}

\subsection{Organization and Social Structuring}

Counting alone was not sufficient. The human mind had a tendency toward order and arrangement: placing things next to each other, comparing and sorting. Which prey was more important, which path was shorter, or which seed should be planted first—all of these required building some kind of "hierarchy" and "order." This capacity for organization was not instinctual, but the product of a cognitive leap that distinguished humans from other beings.\\
Imagine you are the leader of a group of early humans who has thirteen prey at disposal for dividing for evening consumption. He faces an abstract-practical problem: how to allocate these thirteen units to members in the best way so that both justice is observed and the group's survival is guaranteed?\\
This problem requires building hierarchies of priorities, distribution rules, and recording tools (external memory)—the same tools that later appeared in the form of tokens and tablets. The human mind showed an inclination toward order and arrangement: placing things together, comparing and sorting. Which prey is more important, which path is shorter, or which seed should be planted first.\\
These problems demanded the same tools that later appeared in the form of tokens, clay tablets, and ultimately data structures. \textbf{Cognitive organization was a prerequisite for resource management and institutional formation.}

\subsubsection{Key Definitions}

For conceptual clarity, the following definitions are necessary:\\
\textbf{Number Sense}: Intuitive capacity for recognizing small quantities and approximating the value of sets without explicit counting.\\
\textbf{Subitizing}: Immediate recognition of small numbers (usually up to four) without step-by-step counting.\\
\textbf{Cardinality \& Ordinality}: Cardinality expresses the size of a set ("how many?"); Ordinality specifies the position of a member in a sequence ("which one?").\\
\textbf{External Memory}: Using symbols, signs, or objects to store information beyond individual memory—a fundamental prerequisite for data structures.

\subsection{From Perception to Material Representation}

It can be said that humans for the first time saw the world not merely as a habitat, but also as a system—a system that must be measured, counted, and organized. This cognitive transformation is considered a turning point in human history because it laid the foundation for systematic and abstract thinking.\\
If these cognitive and social capacities were real, the important next question is how these abilities were recorded in material objects and signs? In other words, how did humanity's internal number sense connect to "scored bones," "marked stones," and tangible counting systems? This question leads us to examine archaeological evidence and analyze humanity's first attempts to create external memory.\\
This question leads us to examine material and archaeological evidence that demonstrates how these cognitive capacities were embodied in tangible tools and symbols. In continuation, we will examine these developments in the Paleolithic period and their consequences.

\section{Paleolithic Counting: Bones, Stones, and Fingers}
% Ishango bone (20,000 BCE), tally systems, body-part counting
% Archaeological evidence of early systematic representation

\section{Neolithic Revolution: Agriculture and the Need for Records}
% Agricultural surplus, seasonal tracking, early inventory systems
% The transition from nomadic to systematic settlement patterns

\section{Proto-Writing and Symbolic Representation}
% Token systems (8,000 BCE), clay envelopes, early symbolic abstraction
% The cognitive leap from concrete to abstract representation