% ==========================================
% CHAPTER 1: THE PRIMORDIAL URGE TO COUNT AND ORDER
% ==========================================

\chapter{The Primordial Urge to Count and Order}
% Timeline: 100,000 BCE - 8,000 BCE
% Focus: Cognitive foundations of systematic thinking

\section{The Philosophy of Measurement and Human Consciousness}
% Why humans needed to count, measure, track - cognitive foundations
% The emergence of abstract thinking about quantity and order

The fact that the human race has been able to reach the threshold of evolution can mainly be attributed to two abilities: intelligence and creativity. While the former enables us to reason, solve problems, learn from experiences, and think abstractly, the latter enables us to change our way of thinking and create new strategies to overcome the obstacles we face. It can be concluded that every aspect of our lives in relation to innovation and scientific progress can be attributed to intelligence or creativity.\\
Based on this, it can be argued that many aspects of the formation of human computational structures—including arrays—are the result of the interaction of these two capabilities: biological intelligence that makes quantitative recognition possible and creativity that transforms these recognitions into symbols and material tools. This chapter examines how these capacities emerged and transformed into cognitive and cultural tools.

\subsection{Instinct versus Representation: Cognitive Foundations of Human Distinction}

Paleontologists and anthropologists have long been searching for explanations of the roots of modernity and modern thought. These discussions usually involve concepts such as "abstraction" and "symbolic thinking" that are often presented without precise and operational definitions.\\
To understand humanity's need for structured information systems, we must first clarify the distinction between instinctual behavior and the capacity for representation.\\
William James, the prominent 19th-century psychologist, provided a clear definition of instinct:

\begin{quote}
	Instinct is usually defined as the faculty of acting in such a way as to produce certain ends, without foresight of the ends, and without previous education in the performance. That instincts, as thus defined, exist on an enormous scale in the animal kingdom, needs no proof. They are the functional correlatives of structure. With the presence of a certain organ goes, one may say, almost always a native aptitude for its use.
\end{quote}
This definition provides deep insight into the animal kingdom: instincts are hardwired responses that automatically emerge from biological structure. A spider does not need to learn web-weaving techniques—this pattern is encoded in its neural architecture. A bird does not need to be taught to fly—motor coordination naturally emerges from its physical form.\\
But does this definition apply to humans in the same way it applies to other animals? Can we claim that early humans, from birth, had instinctual abilities for counting, organizing, and creating systematic representations of quantity? The answer is decidedly negative. Early humans, unlike animals, were not limited to instinctual reactions alone. They were able to demonstrate something beyond biological response: the capacity for representation. In other words, humans not only "saw" that one prey was larger than another, but could hold this difference in their minds as a "quantity," compare it, and even transmit it to others. This capacity for mental representation and conceptual transmission is considered a turning point in human cognitive evolution.
\subsection{Number Sense: Cognitive Foundations of Quantitative Thinking}

This capacity for representation, now called "number sense," represents humanity's tendency and ability to use numbers and quantitative methods as means for establishing communication, processing, and interpreting information. This leads to the expectation that mathematics possesses a particular order. Cognitive psychology research has shown that most humans—even without formal education—can intuitively perceive the difference between two and three. Toddlers also reveal such distinctions in their simple games. But humanity's strength was that it transformed this primitive sense into an abstract tool: from "more and less" to "exactly how many?"\\
In attempting to identify a prototypical cognitive basis for the origin of modern symbolic thinking, each candidate feature must satisfy four essential criteria:

\begin{enumerate}
	\item \textbf{Innate}: Determined by factors that exist from birth and are genetically inherited
	\item \textbf{Fundamental}: Essential for basic representational structure
	\item \textbf{Shared}: Observable with other species
	\item \textbf{Bridging}: Capable of connecting the world of sensory perception to the realm of concepts and shared expression
\end{enumerate}
Anthropological and cognitive psychology research has shown that number sense possesses these characteristics.\\
This cognitive development, observed in human infants and even some animal species (such as primates, birds, and mammals), demonstrates an innate and shared characteristic. This trait appears in early human development and is genetically inherited. Moreover, this ability forms the foundation of modern symbolic thinking.\\
Most humans, even without formal education, can intuitively perceive the difference between two and three. Children also reveal such distinctions in their simple games. But humanity's strength was that it transformed this primitive sense into an abstract tool: from "more and less" to "exactly how many?" This development represents the transition from intuitive understanding to precise measurement.

\subsection{From Intuitive Understanding to Systematic Measurement}

Imagine returning to ancient times and observing a group of early humans hunting. While hunting, without needing to know mathematics, they would go toward better prey through observation. For example, when facing two deer, one fat and one thin, the natural choice would be toward the fatter animal, because more food would be available for the group's meal. From our perspective, early humans made such decisions.\\
But when two or three fat deer were fleeing from a group of early humans, would they still choose one? The answer is probably negative. They considered these creatures as a quantitative set and would say there were "two" or "three" deer that needed to be hunted.\\
While in comparing the fat and thin deer, they would select the fat one and reject the thin one. This shows that humans from the beginning had two types of quantitative thinking: qualitative comparison (fat versus thin) and absolute counting (two versus three).\\
This is where the concept of "measurement" emerges. Measurement means dividing a continuous reality—such as time, distance, or weight—into discrete units that can be counted again. When early humans learned to count nights, or link the length of seasons with changes in sky and earth, they actually took the first step in building the "language of order." This language later appeared in the form of symbols, numbers, and ultimately tables and arrays.\\
This cognitive-cultural break has deep implications that are still observable in modern data structures:

\begin{itemize}
	\item \textbf{Discretization}: Converting continuity into discrete units
	\item \textbf{Standardization}: Creating fixed and transferable units
	\item \textbf{Reproducibility}: Enabling precise reproduction of information
\end{itemize}

\subsection{Organization and Social Structuring}

Counting alone was not sufficient. The human mind had a tendency toward order and arrangement: placing things next to each other, comparing and sorting. Which prey was more important, which path was shorter, or which seed should be planted first—all of these required building some kind of "hierarchy" and "order." This capacity for organization was not instinctual, but the product of a cognitive leap that distinguished humans from other beings.\\
Imagine you are the leader of a group of early humans who has thirteen prey at disposal for dividing for evening consumption. He faces an abstract-practical problem: how to allocate these thirteen units to members in the best way so that both justice is observed and the group's survival is guaranteed?\\
This problem requires building hierarchies of priorities, distribution rules, and recording tools (external memory)—the same tools that later appeared in the form of tokens and tablets. The human mind showed an inclination toward order and arrangement: placing things together, comparing and sorting. Which prey is more important, which path is shorter, or which seed should be planted first.\\
These problems demanded the same tools that later appeared in the form of tokens, clay tablets, and ultimately data structures. \textbf{Cognitive organization was a prerequisite for resource management and institutional formation.}

\subsubsection{Key Definitions}

For conceptual clarity, the following definitions are necessary:\\
\textbf{Number Sense}: Intuitive capacity for recognizing small quantities and approximating the value of sets without explicit counting.\\
\textbf{Subitizing}: Immediate recognition of small numbers (usually up to four) without step-by-step counting.\\
\textbf{Cardinality \& Ordinality}: Cardinality expresses the size of a set ("how many?"); Ordinality specifies the position of a member in a sequence ("which one?").\\
\textbf{External Memory}: Using symbols, signs, or objects to store information beyond individual memory—a fundamental prerequisite for data structures.

\subsection{From Perception to Material Representation}

It can be said that humans for the first time saw the world not merely as a habitat, but also as a system—a system that must be measured, counted, and organized. This cognitive transformation is considered a turning point in human history because it laid the foundation for systematic and abstract thinking.\\
If these cognitive and social capacities were real, the important next question is how these abilities were recorded in material objects and signs? In other words, how did humanity's internal number sense connect to "scored bones," "marked stones," and tangible counting systems? This question leads us to examine archaeological evidence and analyze humanity's first attempts to create external memory.\\
This question leads us to examine material and archaeological evidence that demonstrates how these cognitive capacities were embodied in tangible tools and symbols. In continuation, we will examine these developments in the Paleolithic period and their consequences.
\newpage
\section{Paleolithic Counting: Bones, Stones, and Fingers}
% Ishango bone (20,000 BCE), tally systems, body-part counting
% Archaeological evidence of early systematic representation
In a narrow but rather imprecise sense, a numeral system is the method by which humans represent numbers. We have already limited our discussion, because among all known species only humans possess the ability to count and form numbers that we can later perform calculations upon. Many numeral systems—often very different—have been used by many cultures and civilizations again highly diverseover the ages, and a wide range of them still exist even today in our relatively global society. In a much broader sense, a numeral system is the set of various ways humans reason about numbers, and that is the definition we will use for our discussion. But what do we mean by the definition above? Well, let us talk about the ways in which we, as humans, reason about numbers. When we speak about numbers we reason about them, so we need a way to represent numbers in speech. When we write about numbers we reason about them, so we need a way to represent numbers in writing/text. This representation is known as notation. Moreover, when reasoning about numbers, we need some kind of base or radix, which is the fundamental number to which all other numbers relate. In recent times—perhaps a few centuries ago—we have also reasoned extensively about the different sets of numbers we have been able to create. Thus, the term numeral system also comes to mean one of these constructed sets.\cite{weibull_numbersystems} \\ 
Figuring out when humans began to count systematically, with purpose, is not easy. Our first real clues are a handful of curious, carved bones dating from the final few millennia of the three-million-year expanse of the Old Stone Age, or Paleolithic era. Those bones are humanity’s first pocket calculators: For the prehistoric humans who carved them, they were mathematical notebooks and counting aids rolled into one. For the anthropologists who unearthed them thousands of years later, they were proof that our ability to count had manifested itself no later than 40,000 years ago.\\
In 1973, while excavating a cave in the Lebombo Mountains, near South Africa’s border with Swaziland, Peter Beaumont found a small, broken bone with twenty-nine notches carved across it. The so-called Border Cave had been known to archaeologists since 1934, but the discovery during World War II of skeletal remains dating to the Middle Stone Age heralded a site of rare importance. It was not until Beaumont’s dig in the 1970s, however, that the cave gave up its most significant treasure: the earliest known tally stick, in the form of a notched, three-inch long baboon fibula.\\
On the face of it, the numerical instrument known as the tally stick is exceedingly mundane. Used since before recorded history—still used, in fact, by some cultures—to mark the passing days, or to account for goods or monies given or received, most tally sticks are no more than wooden rods incised with notches along their length. They help their users to count, to remember, and to transfer ownership. All of which is reminiscent of writing, except that writing did not arrive until a scant 5,000 years ago—and so, when the Lebombo bone was determined to be some 42,000 years old, it instantly became one of the most intriguing archaeological artifacts ever found. Not only does it put a date on when Homo sapiens started counting, it also marks the point at which we began to delegate our memories to external devices, thereby unburdening our minds so that they might be used for something else instead. Writing in 1776, the German historian Justus Möser knew nothing of the Lebombo bone, but his musings on tally sticks in general are strikingly apposite:

\begin{quote}
The notched tally stick itself testifies to the intelligence of our ancestors. No invention is simpler and yet more significant than this.
\end{quote}
It is not clear what quantity the twenty-nine notches carved into the Border Cave’s baboon fibula represents. It is a number, that much is known: had the bone been purely decorative, the notches would have been added all at once, but four different tools were used over time to add to the count. As such, the Lebombo bone is likely to be the earliest mathematical device ever found. (Sadly, it is too great a leap to call it the earliest known pocket calculator. Humans started wearing clothes around 170,000 years ago, but pockets themselves are probably no more than a few thousand years old.)\\
If the Lebombo bone answers the question, at least partly, of when humans learned to count, it leaves another one unanswered: How did they learn to do so?\\
Counting, fundamentally, is the act of assigning distinct labels to each member of a group of similar things to convey either the size of that group or the position of individual items within it. The first type of counting yields cardinal numbers such as "one," "two," and "three"; the second gives ordinals such as "first," "second," and "third."
At first, our hominid ancestors probably did not count very high. Many body parts present themselves in pairs—arms, hands, eyes, ears, and so on—thereby leading to an innate familiarity with the concept of a pair and, by extension, the numbers 1 and 2. But when those hominids regarded the wider world, they did not yet find a need to count much higher. One wolf is manageable; two wolves are a challenge; any more than that and time spent counting wolves is better spent making oneself scarce. The result is that the very smallest whole numbers have a special place in human culture, and especially in language. English, for instance, has a host of specialized terms centered around twoness: a brace of pheasants; a team of horses; a yoke of oxen; a pair of, well, anything. An ancient Greek could employ specific plurals to distinguish between groups of one, two, and many friends (ho philos, to philo, and hoi philoi). In Latin, the numbers 1 to 4 get special treatment, much as “one” and “two” correspond to “first” and “second,” while “three” and “four” correspond directly with “third” and “fourth.” The Romans extended that special treatment into their day-to-day lives: after their first four sons, a Roman family would typically name the rest by number (Quintus, Sextus, Septimus, and so forth), and only the first four months of the early Roman calendar had proper names. Even tally marks, the age-old “five-barred gate” used to score card games or track rounds of drinks, speaks of a deep-seated need to keep things simple.\\
Counting in the prehistoric world would have been intimately bound to the actual, not the abstract. Some languages still bear traces of this: a speaker of Fijian may say \textit{doko} to mean “one hundred mulberry bushes,” but also \textit{koro} to mean “one hundred coconuts.” Germans will talk about a \textit{Faden}, meaning a length of thread about the same width as an adult’s outstretched arms. The Japanese count different kinds of things in different ways: there are separate sequences of cardinal numbers for books; for other bundles of paper such as magazines and newspapers; for cars, appliances, bicycles, and similar machines; for animals and demons; for long, thin objects such as pencils or rivers; for small, round objects; for people; and more.\\
Gradually, as our day-to-day lives took on more structure and sophistication, so, too, did our ability to count. When farming a herd of livestock, for example, keeping track of the number of one’s sheep or goats was of paramount importance, and as humans divided themselves more rigidly into groups of friends and foes, those who could count allies and enemies had an advantage over those who could not. Number words graduated from being labels for physical objects into abstract concepts that floated around in the mental ether until they were assigned to actual things.\\
Even so, we still have no real idea how early humans started to count in the first place. Did they gesture? Speak? Gather pebbles in the correct amount? To form an educated guess, anthropologists have turned to those tribes and peoples isolated from the greater body of humanity, whether by accident of geography or deliberate seclusion. The conclusion they reached is simple. We learned to count with our fingers.\\
A 1913 survey of the number words used by several Native American tribes found that many of those words were related to “finger,” “thumb,” and “hand.” Counterintuitively, perhaps, despite the general possession of ten fingers per person, fewer than half of those tribes counted in multiples of ten. About a third used systems that revolved around the number 5, which was often referred to as “fingers finished,” “all finished,” “gone,” or “spent.” A further tenth of the tribes used vigesimal schemes based on the number 20 (“all hands and feet”), while a few contrarian outliers used 2-, 3-, and 4-based systems with less obvious connections to human anatomy.\\
Fifteen years earlier, a group of scientists from Cambridge, England, had made a series of visits to the islands of the Torres Straits, strung between Papua New Guinea to the north and Australia to the south. A.C. Haddon, the driving force behind the expeditions, recounted:

\begin{quote}
There was another system of counting by commencing at the little finger of the left hand, kotodimura, then following on with the fourth finger, kotodimura gorngozinga (or quruzinger); middle finger, il get; index finger, klak-nětoi-gět; thumb, kabaget; wrist, perta or tiap; elbow joint, kudu; shoulder, zugukwoik; left nipple, susu madu; sternum, kosa, dadir; right nipple, susu madu, and ending with the little finger of the right hand.
\end{quote}
In this way, Haddon said, starting on one side of the body and traversing over to the other, the islanders could count to nineteen. More recently, a math teacher named Glen Lean catalogued the number words for 883 of the 1,200 known languages from Papua New Guinea and Micronesia and found that the use of fingers for counting was foundational to many of those languages. Like the Torres Strait islanders, the Papua New Guineans then carried on to the forearm, elbow, eyes, nose, ears, and other body parts. A study of Yupno, a language indigenous to Papua New Guinea’s Finisterre Mountain range, recorded that Yupno men added their testicles and penis for good measure, allowing them to count to thirty-three using body parts alone.
\\
Hold my earliest attested beer, an ancient Sumerian might have said.\\
From the sixth millennium onward, the valley between the Tigris and the Euphrates Rivers—Mesopotamia, the ancient Greeks called it, the “land between rivers”—harbored one of the world’s earliest civilizations. Having mastered animal husbandry and the cultivation of crops, Mesopotamian farmers became the engine of a new agrarian economy. Almost from the beginning, it seems, they used small clay tokens an inch or so in size, hand-rolled into the shape of spheres, cones, disks, and other simple shapes, to keep records. Each shape stood for a fixed quantity of some good or other: A cone represented a small quantity of cereal, a sphere a larger amount, and a flat disk the largest. Ovoids were jars of oil; cylinders and rounded disks were farm animals; and so on.\\
Around 3300 BC, as Mesopotamia's scattered farming communities began to coalesce into the patchwork of city-states called Sumer, their use of tokens became more sophisticated. At first, batches of tokens were wrapped in clay balls called bullae and marked with personal seals to create records of important transactions. Later, the surfaces of those bullae were impressed with the tokens to be sealed inside so that a bulla's contents could be divined without having to break it open. Once it became apparent that the signs on the outside were as useful as the tokens on the inside, the tokens themselves became surplus to requirements—and the signs, says a theory first proposed by French-American archaeologist Denise Schmandt-Besserat, evolved into the distinctive angular form of cuneiform writing.\\
Cuneiform tablets show that the Sumerians and their successors, the Akkadians and Babylonians, used sexagesimal numbers. That is, their numerical system was rooted in the number 60. Whereas decimal gives rise to round numbers such as 1, 10, and 100 (or 10 squared), the Sumerians counted in terms of 1, 60, 3,600 (or 60 squared), and so on. There are practical advantages to this, since 60 can be divided into whole numbers by 1, 2, 3, 4, 5, 6, 10, 12, 15, 20, 30, and 60, but, as E. F. Robertson, late of St. Andrews University in Scotland, points out, it is rare for a culture to choose the base for its number system. More often, as illustrated by those Native American tribes, it naturally settles upon a base when it begins to count. Counting on five fingers leads to the quinary system, or base 5; two hands lead to decimal, or base 10; two hands and two feet to base 20, or vigesimal. How, then, did the Sumerians land on base 60?\\
The answer lies in the Sumerians' tokens and bullae. Successive scholars have noted that the shapes made when tokens were pushed into the soft clay of a bulla appear to be very similar to the number symbols used on the earliest "proto-literate" clay tablets. That is, the shapes and the values of physical tokens seem to have carried over directly to the written sexagesimal numerals used by the earliest literate Sumerians. As such, the ancient Mesopotamians must have been counting in base 60 on their fingers long before they, or, indeed, anyone else on the planet, could set out numbers in writing.\\
The Mesopotamians' unique counting method is thought to come from a mix of a duodecimal system that used the twelve finger joints of one hand and a quinary system that used the five fingers of the other. By pointing at one of the left hand’s twelve joints with one of the right hand’s five digits, or, perhaps, by counting to twelve with the thumb of one hand and recording multiples of twelve with the digits of the other, it is possible to represent any number from 1 to 60. However it worked, the Mesopotamians’ anatomical calculator was a thing of exceptional elegance, and the numbers they counted with it echo through history. It is no coincidence that a clock has twelve hours, an hour has sixty minutes, and a minute has sixty seconds.\cite{houston2023_earlyhistory}
\newpage

\section{Neolithic Revolution: Agriculture and the Need for Records}
% Agricultural surplus, seasonal tracking, early inventory systems
% The transition from nomadic to systematic settlement patterns

\section{Proto-Writing and Symbolic Representation}
% Token systems (8,000 BCE), clay envelopes, early symbolic abstraction
% The cognitive leap from concrete to abstract representation