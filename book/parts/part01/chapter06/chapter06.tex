% ==========================================
% CHAPTER 6: MAYAN MATHEMATICS AND CALENDAR SYSTEMS
% ==========================================

\chapter{Mayan Mathematics and Calendar Systems}
% Timeline: 2,000 BCE - 1,500 CE
% Focus: Independent development of mathematical concepts in Mesoamerica

\section{Mayan Vigesimal System and Zero Concept}
% Base-20 number system, independent invention of zero
% Shell symbol for zero and positional notation
% Comparison with Old World mathematical development

\section{The Long Count Calendar: Systematic Time Representation}
% Systematic hierarchical time units, positional calendar notation
% Computational complexity of Mayan chronology
% Early concepts of systematic temporal data structures

\section{Mayan Astronomical Tables and Systematic Observation}
% Venus tables, eclipse predictions, systematic astronomical data
% Dresden Codex and systematic mathematical record-keeping
% Early concepts of periodic data and pattern recognition

\section{Architectural Mathematics and Systematic Proportions}
% Pyramid construction, systematic geometric planning
% Mathematical relationships in Mayan architecture
% Evidence of systematic mathematical knowledge application
