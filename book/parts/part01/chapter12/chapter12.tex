% ==========================================
% CHAPTER 12: PERSIAN MATHEMATICAL GENIUS AND SYSTEMATIC INNOVATION
% ==========================================

\chapter{Persian Mathematical Genius and Systematic Innovation}
% Timeline: 800 CE - 1500 CE
% Focus: Persian contributions to mathematics and systematic thinking

\section{Al-Khwarizmi: The Persian Father of Algebra and Algorithms}
% Muhammad ibn Musa al-Khwarizmi from Khwarezm (modern-day Iran/Central Asia)
% Al-Jabr wa-l-Muqābala, systematic solution methods
% The word "algorithm" derived from "al-Khwarizmi"
% Foundational concepts for systematic mathematical procedures

\section{Omar Khayyam: Poet-Mathematician and Geometric Revolutionary}
% Treatise on Demonstration of Problems of Algebra
% Cubic equations solved through systematic geometric methods
% Systematic classification of algebraic equations by degree
% Integration of geometry and algebraic thinking
% Calendar reform and systematic astronomical calculation

\section{Al-Biruni: The Persian Polymath and Systematic Empiricism}
% Abu Rayhan al-Biruni, systematic experimental methods
% Comparative studies across cultures, systematic empirical investigation
% Mathematical geography and systematic measurement techniques
% Early concepts of systematic scientific methodology

\section{Nasir al-Din al-Tusi and Systematic Astronomical Mathematics}
% Tusi-couple and advanced geometric methods
% Systematic astronomical observation and mathematical modeling
% The Tusi couple as precursor to systematic coordinate transformations
% Systematic approaches to celestial mechanics

\section{Persian Computational Instruments and Systematic Calculation}
% Persian astrolabe development, systematic computational aids
% Mechanical calculation devices and systematic astronomical computation
% Integration of theoretical mathematics with practical instruments

\section{Ghiyath al-Din Jamshid Kashani: Systematic Decimal Innovation}
% The Kashani decimal approximation methods
% Systematic approaches to decimal calculations and π approximation
% Advanced systematic computational techniques