% ==========================================
% CHAPTER 4: INDUS VALLEY CIVILIZATION - LOST SYSTEMS OF ORDER
% ==========================================

\chapter{Indus Valley Civilization: Lost Systems of Order}
% Timeline: 3,300 BCE - 1,300 BCE
% Focus: Archaeological evidence of sophisticated systematic thinking

\section{Urban Planning and Systematic Organization}
% Grid-based city layouts, standardized measurements
% Harappa and Mohenjo-daro as examples of systematic planning
% Evidence of systematic administrative and planning systems

\section{The Indus Script Mystery}
% Undeciphered writing system, statistical analysis of symbols
% Evidence of systematic symbolic representation
% Computational approaches to script analysis

\section{Standardization and Systematic Manufacturing}
% Uniform weights and measures, mass production techniques
% Evidence of systematic quality control and standards
% The decimal weight system and measurement precision

\section{Trade Networks and Information Systems}
% Long-distance trade, administrative seals, systematic commerce
% Early evidence of distributed information management
% Systematic approaches to trade record-keeping

\section{Water Management and Systematic Engineering}
% Sophisticated drainage systems, systematic urban engineering
% Evidence of systematic hydraulic planning
% Early concepts of systematic infrastructure management