% ==========================================
% CHAPTER 2: MESOPOTAMIAN FOUNDATIONS OF SYSTEMATIC THINKING
% ==========================================

\chapter{Mesopotamian Foundations of Systematic Thinking}
% Timeline: 3,500 BCE - 1,750 BCE
% Focus: Birth of written mathematics and positional systems

\section{Sumerian Cuneiform and Early Record-Keeping}
% Clay tablets, administrative records, systematic documentation
% The world's first bureaucratic data management systems
% Uruk archives and early database concepts

\section{The Revolutionary Base-60 System}
% Sexagesimal notation, place-value concepts, mathematical implications
% Why base-60 and its influence on modern timekeeping and geometry
% Computational advantages of the sexagesimal system

\section{Babylonian Mathematical Tablets}
% Plimpton 322, YBC 7289, systematic mathematical procedures
% Early algorithmic thinking and tabular arrangements
% Mathematical tables as computational aids

\section{The Concept of Position and Place Value}
% Positional notation development, empty positions (proto-zero)
% Conceptual foundations for array indexing
% The birth of systematic positional representation

\section{Hammurabi's Code: Systematic Legal Data Structures}
% Legal codes as early databases, systematic classification
% Rule-based systems and conditional logic in law
% Early concepts of structured information organization