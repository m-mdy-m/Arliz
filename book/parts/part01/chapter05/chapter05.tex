% ==========================================
% CHAPTER 5: ANCIENT CHINESE MATHEMATICAL MATRICES AND SYSTEMATIC THINKING
% ==========================================

\chapter{Ancient Chinese Mathematical Matrices and Systematic Thinking}
% Timeline: 2,000 BCE - 220 CE
% Focus: Development of matrix concepts and systematic calculation

\section{Oracle Bones and Early Binary Concepts}
% Shang dynasty divination, binary-like symbolic systems
% I Ching hexagrams and systematic symbolic manipulation
% Early concepts of systematic state enumeration

\section{The Nine Chapters on Mathematical Art}
% Systematic problem-solving methods, coefficient arrays
% Early matrix operations for solving linear systems (fangcheng)
% Gaussian elimination in ancient Chinese mathematics

\section{Chinese Rod Numerals and Counting Boards}
% Systematic positional calculation, mechanical computation aids
% Early concepts of state-based calculation systems
% The suanpan abacus and mechanical array operations

\section{Han Dynasty Administrative Mathematics}
% Systematic record-keeping, statistical methods
% Early concepts of data organization and analysis
% The Grand Secretariat and systematic bureaucratic data

\section{Zu Chongzhi and Systematic Approximation Methods}
% π calculations, systematic approximation techniques
% Early concepts of iterative computational methods
% Systematic approaches to mathematical precision
