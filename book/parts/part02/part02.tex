
\part{Mathematical Fundamentals}
\section*{Introduction}

The historical journey in Part 1 showed us how humans developed systematic thinking about organized information. Now we need to translate those insights into the precise mathematical language that makes arrays work.\\
This isn't about learning math for math's sake. Every mathematical concept we explore here—from basic number properties to linear algebra—directly enables the array operations you'll use in programming. When you understand why multiplication is commutative, you'll understand why certain array optimizations work. When you grasp set theory, you'll see the logic behind array search algorithms. When you work with mathematical functions, you'll understand the elegant relationship between array indices and their values.\\
We'll build everything from first principles, assuming no advanced mathematical background. But we won't treat mathematics as a collection of arbitrary rules. Instead, we'll see how each concept emerged from the same human drive for systematic organization that we traced in Part 1.\\
Think of this part as building your mathematical toolkit. Every tool we create here will be used extensively in later parts. By the end, you'll have the mathematical foundation needed to truly understand not just how arrays work, but why they work the way they do.